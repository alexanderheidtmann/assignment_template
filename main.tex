\ProvidesPackage{settings}

\renewcommand{\baselinestretch}{1.25}
\usepackage[T1]{fontenc}
\usepackage[utf8]{inputenc}
\usepackage{lipsum}
\usepackage{lmodern}
\usepackage[english]{babel}
\usepackage{geometry} 
\usepackage{fancyhdr}
\usepackage[dvipsnames]{xcolor}
\usepackage{amsmath,amssymb,amsfonts,amsthm}
\usepackage{bm}
\usepackage{multirow}
\usepackage[Symbol]{upgreek}
\usepackage{graphicx}
\usepackage{booktabs}
\usepackage[framed, numbered]{mcode}
\usepackage{caption}
\usepackage{subcaption}
\usepackage{pgfplots}
\usepackage{listings}
\pgfplotsset{compat=1.13}
\usepackage{tikz}
\usepackage{pgfplotstable}
\usepackage{tikz-3dplot}
\usepackage{MnSymbol,wasysym}
\usepackage{setspace}
\usepackage{epstopdf}
\usepackage{wrapfig}
\usepackage{float}
\usepackage{gensymb}
\usepackage{tabularx}
\usepackage{eurosym}
\usepackage{amstext} % for \text
%\DeclareRobustCommand{\officialeuro}{%
%  \ifmmode\expandafter\text\fi
%  {\fontencoding{U}\fontfamily{eurosym}\selectfont e}}
%\geometry{a4paper,tmargin=2cm, bmargin=2.3cm, lmargin=3cm, rmargin=3cm, headheight=1cm, headsep=0.5cm, footskip=0.5cm}

\pgfplotsset{yticklabel style={text width=3em,align=right}}
\pgfplotsset{every tick label/.append style={font=\tiny}}

\newcommand{\norm}[1]{\left\lVert#1\right\rVert_2}
\newcommand{\matvet}[1]{\bm{\mathrm{#1}}}
\newcommand{\unit}[1]{\, \mathrm{#1}}
\newlength\figureheight
\newlength\figurewidth

\newcommand{\HRule}{\rule{\linewidth}{0.5mm}}
%\newcommand{\namesigdate}[2][5cm]{%
%	\begin{tabular}{@{}p{#1}@{}}
%		#2 \\[1\normalbaselineskip] \hrule \\[-20pt]
%		{\small \textit{Signature}}
%	\end{tabular}
%}   

\fancyhf{} % tom header/footer
\fancyhfoffset{20pt}
\fancyhfoffset{20pt}
\fancyhead[OL]{Wind Tech $\&$ Aerodynamics}
\fancyhead[OR]{Assignment $\#3$ - Structural Mechanics}
\fancyfoot[FL]{}
\fancyfoot[FC]{\thepage}
\fancyfoot[FR]{}
\renewcommand{\headrulewidth}{0.4pt}
\renewcommand{\footrulewidth}{0.4pt}
\pagestyle{fancy}

\definecolor{gray75}{gray}{0.75}
\newcommand{\hsp}{\hspace{20pt}}



\begin{document}
\begin{titlepage}

\begin{center}
\textsc{\LARGE Technical University of Denmark}\\[1.5cm]

\HRule \\
{\Large \bfseries Course \\
\large \bfseries Assignment \\[1cm]
Alexander Heidtmann, s163402 \\
\today \\
\HRule \\[1cm]

\vspace{0.5cm}

\hspace{1cm}
\begin{center}
\end{center}
\hspace{3cm}

\vfill

\noindent
\begin{figure}[h!]
\centering
\includegraphics[width=0.15\textwidth]{setup/DTU.PNG}
\end{figure}
\end{center}
\end{titlepage}


\subsection*{Q\#1}
With the given structural properties for the Tjæreborg Machine also investigated in Assignment \#1, a \textsc{Matlab}-code is developed to calculate the deflection of the wind turbine blade. This is done by implementing the deflection algorithm from Chapter 11 in (Aerodynamics of Wind Turbines, Martin O.L. Hansen, 2nd ed.).\\
To evaluate the results from the algorithm, they are compared to an analytical test, which calculates the displacement with the following equation:

\begin{align*}
u(x) &= \frac{pR^4}{24 EI} (\xi^4 -4\xi +3)\\
\xi &= (R-x)/R
\end{align*}

This analytical approximation is marked in red in the following plot, whereas the deflection of the wind turbine blade is depicted in blue. 

\setlength\figureheight{7cm}
\setlength\figurewidth{13cm}
\begin{figure}[H]
	\hspace*{-1.8cm}
	\centering
	\input{img/q1_plot1.tikz}
	\caption{\small{Numerical simulation versus analytical approximation of the deflection of a wind turbine blade with constant bending stiffness $EI_1$ and $EI_2$ as well as constant loadings, $p$. These are plotted as a function of blade element radius from the hub. The bottommost sub plot shows the displacement in the z-direction.}}
	\label{q1_p1}
\end{figure}

\subsection*{Q\#2}

For this problem, loads data is generated with the BEM-algorithm developed in Assignment \#1. The interesting data are the normal aerodynamic loads $p_N$ and the tangential loads $p_T$. 
The generated data are produced by providing the BEM-algorithm with the following parameters for the three different wind speeds:

\begin{table}[H]
\centering
	\begin{tabular}{c|c c c }
	  & $\theta_p [^\circ]$ & $V_0 [m/s]$ & $C_p$ \\ 
	\hline 
	8 m/s & -2.78 & 8.0 m/s & 0.5437 \\ 
	\hline 
	12 m/s & -2.78 & 12 m/s & 0.4837 \\ 
	\hline 
	20 m/s & 7.00 & 20 m/s & 0.2434 \\ 
	\hline 
	\end{tabular} 
	\caption{\small{Pitch angle and wind speed variables used in BEM-calculation of tangential and normal loads distributions, $p_T$ and $p_N$, respectively. The results of the BEM-calculations yield corresponding power coefficient values, $C_p$.}}
\end{table}

It was found in Assignment \#1, that the highest $C_p$ yield comes with a pitch angle of $-2.78^{\circ}$ with the developed \textsc{Matlab}-code. The tangential and normal loads data are generated with this setting for wind speeds where pitch regulation is not needed. \\
Below are plots of the results for the loads at different wind speeds.

\setlength\figureheight{5cm}
\setlength\figurewidth{13cm}
\begin{figure}[H]
	\hspace*{-1.8cm}
	\centering
	\input{img/q2_nloads.tikz}
	\caption{\small{Normal loads as a function of blade length for the three wind speeds generated by the BEM-algorithm.}}
	\label{q2_nload}
\end{figure}

\setlength\figureheight{5cm}
\setlength\figurewidth{13cm}
\begin{figure}[H]
	\hspace*{-1.8cm}
	\centering
	\input{img/q2_tloads.tikz}
	\caption{\small{Tangential loads as a function of blade length for the three wind speeds generated by the BEM-algorithm.}}
	\label{q2_tload}
\end{figure}
\newpage
On Figure \ref{q2_nload}, normal loads delivered by the BEM-algorithm are plotted for the different wind speeds. It is seen that the lowest wind speed (magenta line) of 8 m/s delivers the smallest loads across the span of the blade. The middle wind speed (red line) of 12 m/s outputs higher loads over the blade, trivially. However, the highest wind speed (blue line) only reaches approximately the same normal loads as the middle wind speed, despite being almost doubled in wind speed. \\
This can be explained by the pitch angle, that is regulated in order to retain rated power output and to protect the rotor. \\
When the blades are pitched, the flap-wise normal direction is turned away from the wind direction, lowering its effective cross section attacked by the incoming wind. This, however, increases the cross section attacking the leading edge of the blade, which is why it is seen in Figure \ref{q2_tload} that the highest wind speed (blue line) is remarkably higher than the two lower wind speeds.


Now that the aerodynamic loads are known, they can be included in the blade deflection algorithm. 

\setlength\figureheight{4.0cm}
\setlength\figurewidth{13cm}
\begin{figure}[H]
	\hspace*{-1.8cm}
	\centering
	\input{img/q2_p1.tikz}
	\caption{\small{Tjæreborg Machine flap-wise(z) numerical blade deflection simulation as a function of blade length for wind speeds 8-, 12-, 20 m/s. Notice how alike the deflections of middle and high wind speeds are.}}
	\label{q2_defl_z}
\end{figure}

\setlength\figureheight{4.0cm}
\setlength\figurewidth{13cm}
\begin{figure}[H]
	\hspace*{-1.8cm}
	\centering
	\input{img/q2_p22.tikz}
	\caption{\small{Tjæreborg Machine flap-wise(y) numerical blade deflection simulation as a function of blade length for wind speeds 8-, 12-, 20 m/s. Notice smaller values compared to deflections in Figure \ref{q2_defl_z}.}}
	\label{q2_defl_y}
\end{figure}

\section*{Q\#3}

To construct the flexibility matrix, a unit load of $1 \unit{N}$ is applied to only one of the points along the span of the cantilever. When the cantilever is exposed to this unit load, the deflection of every point along the span is measured to form the $i$'th column of the flexibility matrix. This is continuously carried out until every deflection of the span has been measured for each unit load position. As an example: $F_{42}$ is the deflection of cantilever element 4 due to the force of a unit magnitude at cantilever point 2. \\
The deflections are calculated using the piecewise relation:

\begin{equation}
F(x) = 
	\begin{cases}
		\frac{p_0}{6EI} (-x^3 + 3ax^2) 	& \text{if } 0 \geq x < a \\
		\frac{p_0}{6EI} a^2 (3x-a) 		& \text{if } a \geq x \geq L,
	\end{cases}
\end{equation}

here, $p_0$ is the unit load, $x$ is the point at which the cantilever deflection is calculated, $a$ is the point from where the unit load is positioned and $L$ is the length of the cantilever beam.\\
In \textsc{Matlab}, the eigenvalues and corresponding eigen frequencies for the system are found in this way:
\begin{lstlisting}
[V, D] 	= eigs(F*M, 4)
V 		= -interp1(r, V, linspace(min(r),max(r),100), 'spline');
omega1 	= sqrt(1/D(1,1))
omega2 	= sqrt(1/D(2,2))
omega3 	= sqrt(1/D(3,3))
omega4 	= sqrt(1/D(4,4))
\end{lstlisting}
The values are interpolated to smooth the curves of the natural frequencies.

\setlength\figureheight{7cm}
\setlength\figurewidth{13cm}
\begin{figure}[H]
	\hspace*{-1.8cm}
	\centering
	\input{img/q3_p1.tikz}
	\caption{\small{This plot shows the first four eigen-modes for the cantilever beam eigenvalue problem.}}
	\label{q3_eigenmodes}
\end{figure}

The eigen frequencies for the plotted natural modes are found to be
\begin{align*}
\omega_1 &= 4.51 \unit{Hz} \\
\omega_2 &= 22.96 \unit{Hz} \\
\omega_3 &= 60.63 \unit{Hz} \\
\omega_4 &= 123.47 \unit{Hz} \\
\end{align*}
When an external load is exerted to the cantilever beam, i.e. aerodynamic in this case, the frequency of blade excitation can coincide with on of these natural frequencies and a condition of resonance can occur. This resonance results in greatly increased excitation of the cantilever in question and can cause damages and failure. Hence, the knowledge of these frequencies is of great importance when designing the blades for a wind turbine.

\section*{Q\#4}
In this problem, the stability of the third-to-last outermost radial wind turbine element is calculated. This is done using the equations from lecture notes "Something on stability and aerodynamic damping p3, p4". \\
For one cycle, the angle from the rotor plane at which the blade element is vibrating, $\xi$, is evaluated from -90 degrees to 90 degrees. The initial relative velocity is $V_{rel_0} = 67$ m/s and the initial flow angle is $\phi_0 = 9^{\circ}$. This is carried out for an angle of attack of both $\alpha_0 = 8,20 ^{\circ}$. The first natural frequency of $\omega_1 = 4.51 \unit{Hz}$ is used to do the calculations.
With the equations 

\begin{align*}
    x(t) &= A \cos(\omega t) \\
    \dot{x}(t) &= -A \omega \sin (\omega t) \\
    F_x &= \frac{1}{2} \rho V_{rel}^2 c (C_l \sin(\phi - \xi) - C_d \cos (\phi - \xi))\\
    W &= \oint F_x \dot{x}(t)\, \mathrm{d}t
\end{align*}

\setlength\figureheight{7cm}
\setlength\figurewidth{13cm}
\begin{figure}[H]
	\hspace*{-1.8cm}
	\centering
	\input{img/q4_p1.tikz}
	\caption{\small{Force in Newton as a function of vibration angle $\xi$ for the two cases of angle of attack.}}
	\label{q4_p1}
\end{figure}

The results for the forces are non-dimensionalized with the work:

\setlength\figureheight{7cm}
\setlength\figurewidth{13cm}
\begin{figure}[H]
	\hspace*{-1.7cm}
	\centering
	\input{img/q4_p2.tikz}
	\caption{\small{Absolute, dimensionless forces as a function of vibration angle $\xi$ for the two cases of angle of attack.}}
	\label{q4_p2}
\end{figure}

The work figures are for the two situations as follows:

\begin{align*}
    W_8 &= -9835.9 \unit{J} \\
    W_{20} &= -2330.4 \unit{J}
\end{align*}

\newpage
Evidently, the situation where the angle of attack is 8 degrees, the work output is more negative than the situation with the higher angle of attack. When the angle of attack is lower, the angle between the chord line and the relative wind direction is lower. This means a higher load on the flap-wise span an thus laminar airflow is disrupted and will be converted to turbulent flow instead.\\
The mentioned increased aerodynamic load on the flap-wise span is a possible reason why the work calculated is more negative than the situation with higher angle of attack. When the work is negative, the system is negatively damped and can possibly be unstable for the natural frequency $\omega_1 = 4.51 \unit{Hz}$.\\

It can be seen in Figure \ref{q4_p2}, that the red curve (aoa = 8 degrees) is shifted along the x-axis, relative to the blue curve (aoa = 20 degrees). This is due to the shift in the direction of which the vibration is found. \\
Most notably, however, is for which vibration direction $\xi$, the greatest required force is situated. It is seen that this is where $\xi$ is lowest and highest, respectively. This can be explained by the fact that the z-direction, which is the flap-wise orientation of the blade, is approximately normal to the chord line, and along this span, obviously, the stiffness is lowest. In other words; it is easier to displace the blade elements in the z-direction than in the y-direction, visualized in Figure \ref{q2_defl_z} and \ref{q2_defl_y}, which is exactly what was gathered in Q\#1 and Q\#2.








\end{document}
